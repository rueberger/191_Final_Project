\title{Quandle Colorings on $(r,q)$-torus knots}
\author{
        Andrew Berger \\
	Brandon Flannery \\
	Christopher Sumnicht \\	
}
\date{\today}

\documentclass[12pt]{article}

\begin{document}
\maketitle

\begin{abstract}
	Quandles are an algebraic structure equipped with axioms that mirror all the Reidemeister Moves. As such they play important role in knot theory. The dihedral quandle, for instance is used to compute the Fox $n$-coloring (or $Z_p$ coloring) of a knot. The dihedral quandle is a special case of what is known as the Alexander Quandle. Furthermore, the Alexander Quandle is used to compute a more generalized version of For $n$-coloring known as Quandle Coloring (or $Z_{n,q}$ coloring). In this paper we devise methods of efficiently determining whether a $(r,q)$-torus knot is $Z_{n,q}$ colorable.
\end{abstract}

\section{Introduction}
This is time for all good men to come to the aid of their party!

\paragraph{Outline}
The remainder of this article is organized as follows.
Section~\ref{previous work} gives account of previous work.
Our new and exciting results are described in Section~\ref{results}.
Finally, Section~\ref{conclusions} gives the conclusions.

\section{Previous work}\label{previous work}
A much longer \LaTeXe{} example was written by Gil~\cite{Gil:02}.

\section{Results}\label{results}
In this section we describe the results.

\section{Conclusions}\label{conclusions}
We worked hard, and achieved very little.

\bibliographystyle{abbrv}
\bibliography{main}

\end{document}
