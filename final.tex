
\title{Quandle Colorings on $(r,q)$-torus knots}
\author{
        Andrew Berger \\
	Brandon Flannery \\
	Christopher Sumnicht \\	
}
\date{\today}

\documentclass[12pt]{article}
\usepackage{amsmath, amsfonts}


\begin{document}

\maketitle

\begin{abstract}
	Quandles are an algebraic structure equipped with axioms that mirror all the Reidemeister Moves. As such they play important role in knot theory. The dihedral quandle, for instance is used to compute the Fox $n$-coloring (or $Z_p$ coloring) of a knot. The dihedral quandle is a special case of what is known as the Alexander Quandle. Furthermore, the Alexander Quandle is used to compute a more generalized version of For $n$-coloring known as Quandle Coloring (or $Z_{n,q}$ coloring). In this paper we devise methods of efficiently determining whether a $(r,q)$-torus knot is $Z_{n,q}$ colorable.
\end{abstract}

\section{Introduction}
\subsection{Involutary Quandle}
In this paper we concern ourselves with \textbf{Involutory Quandles}. An \textbf{involutory quandle} is any set $K$ equiped with a binary operation $\cdot$ that satisfies $3$ axioms:

\begin{equation}
	x \cdot x = x
\end{equation}
\begin{equation}
	x \cdot (x \cdot y) = y \\
\end{equation}
\begin{equation}
	x \cdot (y \cdot z) = (x \cdot y) \cdot (x \cdot z) \\
\end{equation}

These equations can be though of a symbolic representations of Reidemeister moves. $(1)$ corresponds to the Type I Reidemeister move, $(2)$ corresponds to the Type II Reidemeister move, and $(3)$ corresponds to the Type III Reidemeister move. (See Fig)

\subsection{Alexander Quandle}

The \textbf{Alexander Quandle} is defined as:

$$ x \cdot y = ty + (1 - t)a $$

$t$ is usually a free variable in $\mathbb{C}$ however for this paper we consider a special case of the Alexander Quandle:

$$ x \cdot y = qy + (1 - q)a $$

where $q$ is a free variable in $\mathbb{Z}_n$. The reasons for this will be made clear when we consider the relationship between the Alexander Quandle and coloring.

\subsection{Dihedral Quandle}
The \textbf{Dihedral Quandle} is defined as:

$$x\cdot y = 2x - y$$

it is a special case of the Alexander quandle when $t$ is evaluated at $-1$.

\subsection{Coloring}

We say that a knot $K$ is $Z_p$ colorable if given an prime $p > 2$ every strand in the projection of $K$ can be labeled using numbers $0$ to $p-1$, with at least 2 of the labels distinct so that at each crossing we have:

$$ 2x - y - z = 0 \mod p $$

(TODO: Cite FinalPaper.pdf)

\subsection{Quandle Coloring}

(TODO: Complete the description)
We say that a knot is $Z_{q,n}$ if for some unit $q \in \mathbb{Z}$ was a labeling of strands so that

\paragraph{Outline}
The remainder of this article is organized as follows.
Section~\ref{previous work} gives account of previous work.
Our new and exciting results are described in Section~\ref{results}.
Finally, Section~\ref{conclusions} gives the conclusions.

\section{Previous work}\label{previous work}
A much longer \LaTeXe{} example was written by Gil~\cite{Gil:02}.

\section{Results}\label{results}
In this section we describe the results.

\section{Conclusions}\label{conclusions}
We worked hard, and achieved very little.

\bibliographystyle{abbrv}
\bibliography{main}

\end{document}
