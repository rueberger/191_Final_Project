\documentclass[paper.tex]{subfiles}

\begin{document}

\section{Quandle Coloring of the (2,q)-torus knot}\label{sec:2ntorus}
We first restrict ourselves to $(r,n)$-torus where $r = 2$. Furthermore we only concern $Z_p$ when $p > 2$ is prime to ensure that every element of $Z_p$ is a unit and every so every element has an inverse.

The $(2,n)$-torus knot satisfies the following recurrence relation:
\begin{align*}
	a_0 &= x \\
	a_1 &= y \\
	a_{n+2} &= a_{n} \uc a_{n+1} \\
\end{align*}

We can then substitute the Alexander quandle to conclude:

$$ a_{n+2} = qa_{n} + (1-q)b_{n+1} $$

Solving for $a_n$ we get:

$$ a_n = \frac{(-1)^nqx+t^nx+(-1)^{n+1}y+q^ny}{1+q} $$

Because we restrict ourselves to $p$ being prime division by $1+q$ is allowed.

To determine whether a knot is $\mathbb{Z}_{p,q}$ colorable we must solve for the system of equations so that $\forall x,y \in \mathbb{Z}_n$ the following holds:

\begin{align*}
	a_0 &\equiv a_{n}  & \mod{p} \\
	a_1 &\equiv a_{n+1} & \mod{p} \\
\end{align*}

We now know an explicit formula for $a_n$ so we may substitute it in accordingly:

\begin{align*}
	x &\equiv \frac{(-1)^nqx+q^nx+(-1)^{n+1}y+q^ny}{1+q} & \mod{p} \\
	y &\equiv \frac{(-1)^{n+1}qx+q^{n+1}x+(-1)^{n+2}y+q^{n+1}y}{1+q} & \mod{p} \\
\end{align*}

From another representation of $a_{n}$ we can calculate an alternative equivalence relation

$$a_{0} = a_{p}, a_{1} = a_{p+1}$$
$$x = a_{p-2} + (1-q)a_{p-1}$$
$$ y = a_{p-1} + (1-q)a_{p}$$

To calculate $a_{p}$, we must find the matrix representation of 
$$ a_{n} = (q)a_{n-2} + (1-q)a_{n-1} $$



\[ 
\left[ \begin{array}{ccc}
a_{n} \\
a_{n-1} \\
\end{array} \right] 
=
\left[ \begin{array}{ccc}
q & q-1 \\
0 & 1 \\
\end{array} \right]
\left[ \begin{array}{ccc}
 a_{n-2}\\
a_{n-1} \\
\end{array} \right]
\] 

\[ 
\left[ \begin{array}{ccc}
a_{1} \\
a_{0} \\
\end{array} \right] 
=
\left[ \begin{array}{ccc}
y \\
x \\
\end{array} \right]
\] 

\[ 
\left[ \begin{array}{ccc}
a_{n} \\
a_{n-1} \\
\end{array} \right] 
=
\left[ \begin{array}{ccc}
q-1 & q \\
1 & 0 \\
\end{array} \right] ^{n-1}
\left[ \begin{array}{ccc}
 a_{n-1}\\
a_{n-2} \\
\end{array} \right]
\] 

Thus:

\[ 
\left[ \begin{array}{ccc}
y \\
x \\
\end{array} \right] 
=
\left[ \begin{array}{ccc}
a_{p+1} \\
a_{p} \\
\end{array} \right]
\] 

\[ 
\left[ \begin{array}{ccc}
y \\
x \\
\end{array} \right] 
=
\left[ \begin{array}{ccc}
q & q-1 \\
0 & 1 \\
\end{array} \right] ^{p}
\left[ \begin{array}{ccc}
y \\
x \\
\end{array} \right]
\] 

\[ 
\left[ \begin{array}{ccc}
y \\
x \\
\end{array} \right] 
=
A^{p}
\left[ \begin{array}{ccc}
y \\
x \\
\end{array} \right]
\] 


\[ A^{p} = 
\left[ \begin{array}{ccc}
q-1 & q \\
1 & 0 \\
\end{array} \right] ^{p}
=
PD^{p}P^{-1}
\] 
\[ A = 
\left[ \begin{array}{ccc}
q-1 & q \\
1 & 0 \\
\end{array} \right]
=
PDP^{-1}
\] 
\[ P = 
\left[ \begin{array}{ccc}
-1 & q \\
1 & 1 \\
\end{array} \right]
, P^{-1} =
\left[ \begin{array}{ccc}
-(q+1)^{-1} & q(q+1)^{-1} \\
(q+1)^{-1} & (q+1)^{-1} \\
\end{array} \right]
, D = 
\left[ \begin{array}{ccc}
-1 & 0 \\
0 & q \\
\end{array} \right]
\] 
\[ A^{p}
=
\left[ \begin{array}{ccc}
-1 & q \\
1 & 1 \\
\end{array} \right]
\left[ \begin{array}{ccc}
-1 & 0 \\
0 & q \\
\end{array} \right] ^{p}
\left[ \begin{array}{ccc}
-(q+1)^{-1} & q(q+1)^{-1} \\
(q+1)^{-1} & (q+1)^{-1} \\
\end{array} \right]
\]
\[ A^{p}
=
\left[ \begin{array}{ccc}
-1 & q \\
1 & 1 \\
\end{array} \right]
\left[ \begin{array}{ccc}
-1 & 0 \\
0 & q^{p} \\
\end{array} \right] 
\left[ \begin{array}{ccc}
-(q+1)^{-1} & q(q+1)^{-1} \\
(q+1)^{-1} & (q+1)^{-1} \\
\end{array} \right]
\]

\[
A^{p}=
\left[ \begin{array}{ccc}
\frac{q^{p+1} + (-1)^{p+1}}{q+1} & \frac{q^{p+1} + (-1)^{p}(q)}{q+1} \\
\frac{q^{p} + (-1)^{p+1}}{q+1} & \frac{q^{p} + (-1)^{p}(q)}{q+1} \\
\end{array} \right]
\]

\[
\left[ \begin{array}{ccc}
y \\
x \\
\end{array} \right]=
A^{p}
\left[ \begin{array}{ccc}
y \\
x \\
\end{array} \right]
\]

\[
\left[ \begin{array}{ccc}
y \\
x \\
\end{array} \right]=
\left[ \begin{array}{ccc}
y \frac{q^{p+1} + (-1)^{p+1}}{q+1} + x \frac{q^{p+1} + (-1)^{p}(q)}{q+1} \\
y \frac{q^{p} + (-1)^{p+1}}{q+1} + x \frac{q^{p} + (-1)^{p}(q)}{q+1} \\
\end{array} \right]
\]

Which confirms

$$x = \frac{(-1)^nqx+q^nx+(-1)^{n+1}y+q^ny}{1+q}mod(p) $$

Because we are restricting ourselves to the quandle colorings of knots $n$ must be odd. We can then simplify the congruences to the following:

\begin{align*}
	x &\equiv \frac{-qx+q^nx+y+q^ny}{1+q} & \mod{p} \\
	y &\equiv \frac{qx+q^{n+1}x-y+q^{n+1}y}{1+q} & \mod{p} \\
\end{align*}

From this point, we can arrive upon the $Z_{p,q}$ quandle colorings of any $T(2, p)$ torus knot. Find all $q$  satisfying the equivalence relation above $\forall x,y \in \mathbb{Z}_n$

\section{Quandle Coloring of the (2,q)-torus knot}\label{sec:2ntorus}
We first restricted ourselves to $T(2,p)$-torus knots. Conveniently, we were able to derive an easily diagonalizable recurrence relation matrix, $A$. Now, we are going to try and extend this $Z_{p,q}$ quandle coloring to all $T(p,3)$ knots. 

Remember, the $(2,n)$-torus knot satisfies the following recurrence relation:
\begin{align*}
	a_0 &= x \\
	a_1 &= y \\
	a_{n+2} &= a_{n} \cdot a_{n+1} \\
\end{align*}

Using the braid word of a $T(3,p)$-torus knot, we identified the quandle coloring as satisfying the following recurrence relation. 
\begin{align*}
	a_0 &= x \\
	a_1 &= y \\
	a_2 &= z \\
\end{align*}
\begin{align*}
	a_{n} &= a_{n-3} \triangleleft a_{n-2} \\
	a_{n-1} &= a_{n-4} \triangleleft a_{n-2} \\
	0 &= (n)mod(2)
\end{align*}

Here, there recurrence relation for odd and even values of $n$ are not equivalent. Thus, the recurrence relation matrix, $A$, for the $T(3, p)$ Alexander Quandle has two forms. $A_{0}$ for calculating $a_{n}$ when n $mod(2) = 0$.  $A_{1}$ is for calculating $a_{n}$ when n $mod(2) = 1$:

\[ A_{0} = 
\left[ \begin{array}{ccc}
q-1 & q & 0 \\
q-1 & 0 & q \\
1 & 0 & 0 \\
\end{array} \right]
\]
\[ A_{1} = 
\left[ \begin{array}{ccc}
q-1 & 0 & q \\
q-1 & q & 0 \\
1 & 0 & 0 \\
\end{array} \right]
\]

If $(n)mod(2) = 0$:

\[ 
\left[ \begin{array}{ccc}
a_{n} \\
a_{n-1} \\
a_{n-2} \\
\end{array} \right] 
=
\left[ \begin{array}{ccc}
q-1 & q & 0 \\
q-1 & 0 & q \\
1 & 0 & 0 \\
\end{array} \right]
\left[ \begin{array}{ccc}
a_{n-2} \\
a_{n-3} \\
a_{n-4} \\
\end{array} \right] 
\] 

If $(n)mod(2) = 1$:

\[ 
\left[ \begin{array}{ccc}
a_{n} \\
a_{n-1} \\
a_{n-2} \\
\end{array} \right] 
=
\left[ \begin{array}{ccc}
q-1 & 0 & q \\
q-1 & q & 0 \\
1 & 0 & 0 \\
\end{array} \right]
\left[ \begin{array}{ccc}
a_{n-2} \\
a_{n-3} \\
a_{n-4} \\
\end{array} \right] 
\] 


We can start calculating successive entries in the sequence {$a_{n}$}:

\[ 
\left[ \begin{array}{ccc}
a_{3} \\
a_{2} \\
a_{1} \\
\end{array} \right] 
=
\left[ \begin{array}{ccc}
q-1 & 0 & q \\
q-1 & q & 0 \\
1 & 0 & 0 \\
\end{array} \right]
\left[ \begin{array}{ccc}
z \\
y \\
x \\
\end{array} \right] 
\] 

\[ 
\left[ \begin{array}{ccc}
a_{4} \\
a_{3} \\
a_{2} \\
\end{array} \right] 
=
\left[ \begin{array}{ccc}
q-1 & q & 0 \\
q-1 & 0 & q \\
1 & 0 & 0 \\
\end{array} \right]
\left[ \begin{array}{ccc}
q-1 & 0 & q \\
q-1 & q & 0 \\
1 & 0 & 0 \\
\end{array} \right]
\left[ \begin{array}{ccc}
z \\
y \\
x \\
\end{array} \right] 
\] 

\[ 
\left[ \begin{array}{ccc}
a_{n} \\
a_{n-1} \\
a_{n-2} \\
\end{array} \right] 
=
\left[ \begin{array}{ccc}
q-1 & q & 0 \\
q-1 & 0 & q \\
1 & 0 & 0 \\
\end{array} \right] ^{n//2} 
\left[ \begin{array}{ccc}
q-1 & 0 & q \\
q-1 & q & 0 \\
1 & 0 & 0 \\
\end{array} \right] ^{(n//2) + n mod(2)}
\left[ \begin{array}{ccc}
z \\
y \\
x \\
\end{array} \right] 
\] 

\[ 
\left[ \begin{array}{ccc}
a_{n} \\
a_{n-1} \\
a_{n-2} \\
\end{array} \right] 
=
\left[ \begin{array}{ccc}
q-1 & 0 & q \\
q-1 & q & 0 \\
1 & 0 & 0 \\
\end{array} \right] ^{(n)mod(p)}
\left[ \begin{array}{ccc}
(q-1)^{2} + q(q-1) & q^{2} & (q-1)q \\
(q-1)^{2} + q & 0 & (q-1)q \\
q-1 & 0 & q \\
\end{array} \right] ^{n//2} 
\left[ \begin{array}{ccc}
z \\
y \\
x \\
\end{array} \right] 
\] 

Replicating the calculations of the $T(p,2)$ Alexander quandle equivalence relations, you can find the following equivalence relations:

\begin{align*}
	x \equiv P(x,y,z,q) & \mod{p} \\
	y \equiv P(x,y,z,q) & \mod{p} \\
	z \equiv P(x,y,z,q) & \mod{p} \\
\end{align*}

Given the complexity of the diagonal of $A^{p}$, we were unable to find a simplified form for 

\[ A^{n}
=
\left[ \begin{array}{ccc}
q-1 & 0 & q \\
q-1 & q & 0 \\
1 & 0 & 0 \\
\end{array} \right] ^{(n)mod(2)}
\left[ \begin{array}{ccc}
(q-1)^{2} + q(q-1) & q^{2} & (q-1)q \\
(q-1)^{2} + q & 0 & (q-1)q \\
q-1 & 0 & q \\
\end{array} \right] ^{n//2} 
\]

We suspect there are alternative or brute force methods available to solve the $T(p,3)$ Alexander quandle equivalence relations.  We have not calculated the explicit solution in terms of $q$ for this equivalence set due to computational capacity. Additional endeavors in efficient computing or decomposition of $A^{n}$ are needed to find a timely solution to the equivalence set for the $T(p,3)$ Alexander quandle representations. 


\section{Conclusions: Quandle Coloring of the T(p,q)-torus knot}\label{sec:2ntorus}
Although further calculations can be done to solve for $q$ in 
\begin{align*}
	x \equiv P(x,y,z,q) & \mod{p} \\
	y \equiv P(x,y,z,q) & \mod{p} \\
	z \equiv P(x,y,z,q) & \mod{p} \\
\end{align*}
we were able to define a process for finding the $Z_{p,q}$ Quandle Colorings of T(p, 3) and T(p,2) torus knots. By (1) finding the recurrence relation for quandle labeling of a T(p,q) torus knot, (2) representing the recurrence relation in terms of the alexander quandle, (3) constructiing a recurrence relation matrix,  (4) setting the quandle labeling equivalence relations ($a_{0} \equiv P(a_{0},...,a_{p},q) \mod{p}$) , and (5) solving for $q$ across $\forall x,y \in \mathbb{Z}_n$, we have defined a reasonable process for finding the $Z_{p,q}$ Quandle Colorings of $T(p,q)$ torus knots. 

\section{Further Questions}\label{fqs}

\begin{enumerate}
	\item Consider what happens to $Z_{p,q}$ when $p$ is not prime.
\end{enumerate}

\bibliographystyle{abbrv}
\bibliography{main}

\end{document}
