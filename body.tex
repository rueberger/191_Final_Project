\documentclass[paper.tex]{subfiles}

\begin{document}

\section{Quandle Coloring of the (2,q)-torus knot}\label{sec:2ntorus}
We first restrict ourselves to $(r,n)$-torus where $r = 2$. Furthermore we only concern $Z_p$ when $p > 2$ is prime to ensure that every element of $Z_p$ is a unit and every so every element has an inverse.

The $(2,n)$-torus knot satisfies the following recurrence relation:
\begin{align*}
	a_0 &= x \\
	a_1 &= y \\
	a_{n+2} &= a_{n} \cdot a_{n+1} \\
\end{align*}

We can then substitute the Alexander quandle to conclude:

$$ a_{n+2} = qa_{n} + (1-q)b_{n+1} $$

Solving for $a_n$ we get:

$$ a_n = \frac{(-1)^nqx+t^nx+(-1)^{n+1}y+q^ny}{1+q} $$

Because we restrict ourselves to $p$ being prime division by $1+q$ is allowed.

To determine whether a knot is $\mathbb{Z}_{p,q}$ colorable we must solve for the system of equations so that $\forall x,y \in \mathbb{Z}_n$ the following holds:

\begin{align*}
	a_0 &\equiv a_{n}  & \mod{p} \\
	a_1 &\equiv a_{n+1} & \mod{p} \\
\end{align*}

We now know an explicit formula for $a_n$ so we may substitute it in accordingly:

\begin{align*}
	x &\equiv \frac{(-1)^nqx+q^nx+(-1)^{n+1}y+q^ny}{1+q} & \mod{p} \\
	y &\equiv \frac{(-1)^{n+1}qx+q^{n+1}x+(-1)^{n+2}y+q^{n+1}y}{1+q} & \mod{p} \\
\end{align*}

From another representation of $a_{n}$ we can calculate an alternative equivalence relation

$$a_{0} = a_{p}, a_{1} = a_{p+1}$$
$$x = a_{p-2} + (1-q)a_{p-1}$$
$$ y = a_{p-1} + (1-q)a_{p}$$

To calculate $a_{p}$, we must find the matrix representation of 
$$ a_{n} = (q)a_{n-2} + (1-q)a_{n-1} $$



\[ 
\left[ \begin{array}{ccc}
a_{n} \\
a_{n-1} \\
\end{array} \right] 
=
\left[ \begin{array}{ccc}
q & q-1 \\
0 & 1 \\
\end{array} \right]
\left[ \begin{array}{ccc}
 a_{n-2}\\
a_{n-1} \\
\end{array} \right]
\] 

\[ 
\left[ \begin{array}{ccc}
a_{1} \\
a_{0} \\
\end{array} \right] 
=
\left[ \begin{array}{ccc}
y \\
x \\
\end{array} \right]
\] 

\[ 
\left[ \begin{array}{ccc}
a_{n} \\
a_{n-1} \\
\end{array} \right] 
=
\left[ \begin{array}{ccc}
q-1 & q \\
1 & 0 \\
\end{array} \right] ^{n-1}
\left[ \begin{array}{ccc}
 a_{n-1}\\
a_{n-2} \\
\end{array} \right]
\] 

Thus:

\[ 
\left[ \begin{array}{ccc}
y \\
x \\
\end{array} \right] 
=
\left[ \begin{array}{ccc}
a_{p+1} \\
a_{p} \\
\end{array} \right]
\] 

\[ 
\left[ \begin{array}{ccc}
y \\
x \\
\end{array} \right] 
=
\left[ \begin{array}{ccc}
q & q-1 \\
0 & 1 \\
\end{array} \right] ^{p}
\left[ \begin{array}{ccc}
y \\
x \\
\end{array} \right]
\] 

\[ 
\left[ \begin{array}{ccc}
y \\
x \\
\end{array} \right] 
=
A^{p}
\left[ \begin{array}{ccc}
y \\
x \\
\end{array} \right]
\] 


\[ A^{p} = 
\left[ \begin{array}{ccc}
q-1 & q \\
1 & 0 \\
\end{array} \right] ^{p}
=
PD^{p}P^{-1}
\] 
\[ A = 
\left[ \begin{array}{ccc}
q-1 & q \\
1 & 0 \\
\end{array} \right]
=
PDP^{-1}
\] 
\[ P = 
\left[ \begin{array}{ccc}
-1 & q \\
1 & 1 \\
\end{array} \right]
, P^{-1} =
\left[ \begin{array}{ccc}
-(q+1)^{-1} & q(q+1)^{-1} \\
(q+1)^{-1} & (q+1)^{-1} \\
\end{array} \right]
, D = 
\left[ \begin{array}{ccc}
-1 & 0 \\
0 & q \\
\end{array} \right]
\] 
\[ A^{p}
=
\left[ \begin{array}{ccc}
-1 & q \\
1 & 1 \\
\end{array} \right]
\left[ \begin{array}{ccc}
-1 & 0 \\
0 & q \\
\end{array} \right] ^{p}
\left[ \begin{array}{ccc}
-(q+1)^{-1} & q(q+1)^{-1} \\
(q+1)^{-1} & (q+1)^{-1} \\
\end{array} \right]
\]
\[ A^{p}
=
\left[ \begin{array}{ccc}
-1 & q \\
1 & 1 \\
\end{array} \right]
\left[ \begin{array}{ccc}
-1 & 0 \\
0 & q^{p} \\
\end{array} \right] 
\left[ \begin{array}{ccc}
-(q+1)^{-1} & q(q+1)^{-1} \\
(q+1)^{-1} & (q+1)^{-1} \\
\end{array} \right]
\]

\[
A^{p}=
\left[ \begin{array}{ccc}
\frac{q^{p+1} + (-1)^{p+1}}{q+1} & \frac{q^{p+1} + (-1)^{p}(q)}{q+1} \\
\frac{q^{p} + (-1)^{p+1}}{q+1} & \frac{q^{p} + (-1)^{p}(q)}{q+1} \\
\end{array} \right]
\]

\[
\left[ \begin{array}{ccc}
y \\
x \\
\end{array} \right]=
A^{p}
\left[ \begin{array}{ccc}
y \\
x \\
\end{array} \right]
\]

\[
\left[ \begin{array}{ccc}
y \\
x \\
\end{array} \right]=
\left[ \begin{array}{ccc}
y \frac{q^{p+1} + (-1)^{p+1}}{q+1} + x \frac{q^{p+1} + (-1)^{p}(q)}{q+1} \\
y \frac{q^{p} + (-1)^{p+1}}{q+1} + x \frac{q^{p} + (-1)^{p}(q)}{q+1} \\
\end{array} \right]
\]

Which confirms

$$x = \frac{(-1)^nqx+q^nx+(-1)^{n+1}y+q^ny}{1+q}mod(p) $$

Because we are restricting ourselves to the quandle colorings of knots $n$ must be odd. We can then simplify the congruences to the following:

\begin{align*}
	x &\equiv \frac{-qx+q^nx+y+q^ny}{1+q} & \mod{p} \\
	y &\equiv \frac{qx+q^{n+1}x-y+q^{n+1}y}{1+q} & \mod{p} \\
\end{align*}

\section{Further Questions}\label{fqs}

\begin{enumerate}
	\item Consider what happens to $Z_{p,q}$ when $p$ is not prime.
\end{enumerate}

\bibliographystyle{abbrv}
\bibliography{main}

\end{document}
