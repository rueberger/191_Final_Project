\documentclass[paper.tex]{subfiles}

\begin{document}

\section{Quandle Coloring of the (2,q)-torus knot}\label{sec:2ntorus}
We first restrict ourselves to $(r,n)$-torus where $r = 2$. Furthermore we only concern $Z_p$ when $p > 2$ is prime to ensure that every element of $Z_p$ is a unit and every so every element has an inverse.

The $(2,n)$-torus knot satisfies the following recurrence relation:
\begin{align*}
	a_0 &= x \\
	a_1 &= y \\
	a_{n+2} &= a_{n} \cdot a_{n+1} \\
\end{align*}

We can then substitute the Alexander quandle to conclude:

$$ a_{n+2} = qa_{n} + (1-q)b_{n+1} $$

Solving for $a_n$ we get:

$$ a_n = \frac{(-1)^nqx+t^nx+(-1)^{n+1}y+q^ny}{1+q} $$

Because we restrict ourselves to $p$ being prime division by $1+q$ is allowed.

To determine whether a knot is $\mathbb{Z}_{p,q}$ colorable we must solve for the system of equations so that $\forall x,y \in \mathbb{Z}_n$ the following holds:

\begin{align*}
	a_0 &\equiv a_{n}  & \mod{p} \\
	a_1 &\equiv a_{n+1} & \mod{p} \\
\end{align*}

We now know an explicit formula for $a_n$ so we may substitute it in accordingly:

\begin{align*}
	x &\equiv \frac{(-1)^nqx+q^nx+(-1)^{n+1}y+q^ny}{1+q} & \mod{p} \\
	y &\equiv \frac{(-1)^{n+1}qx+q^{n+1}x+(-1)^{n+2}y+q^{n+1}y}{1+q} & \mod{p} \\
\end{align*}

Because we are restricting ourselves to the quandle colorings of knots $n$ must be odd. We can then simplify the congruences to the following:

\begin{align*}
	x &\equiv \frac{-qx+q^nx+y+q^ny}{1+q} & \mod{p} \\
	y &\equiv \frac{qx+q^{n+1}x-y+q^{n+1}y}{1+q} & \mod{p} \\
\end{align*}

\section{Further Questions}\label{fqs}

\begin{enumerate}
	\item Consider what happens to $Z_{p,q}$ when $p$ is not prime.
\end{enumerate}

\bibliographystyle{abbrv}
\bibliography{main}

\end{document}
